% ----------------------- TODO ---------------------------
% Diese Daten müssen pro Blatt angepasst werden:
\newcommand{\NUMBER}{1}
\newcommand{\EXERCISES}{5}
% Diese Daten müssen einmalig pro Vorlesung angepasst werden:
\newcommand{\COURSE}{Algebra 1}
\newcommand{\TUTOR}{Hauck}
\newcommand{\STUDENTA}{Gounden}
\newcommand{\STUDENTB}{Marttinen}
\newcommand{\STUDENTC}{}
\newcommand{\TODAY}{17.04.2025}
\newcommand{\DEADLINE}{18.04.2025}
% ----------------------- TODO ---------------------------

\documentclass[a4paper]{scrartcl}
\usepackage[utf8]{inputenc}
\usepackage{amsmath}
\usepackage{amssymb}
\usepackage{fancyhdr}
\usepackage{color}
\usepackage{graphicx}
\usepackage{lastpage}
\usepackage{listings}
\usepackage{tikz}
\usepackage{pdflscape}
\usepackage{subfigure}
\usepackage{float}
\usepackage{polynom}
\usepackage{hyperref}
\usepackage{tabularx}
\usepackage{forloop}
\usepackage{geometry}
\usepackage{listings}
\usepackage{fancybox}
\usepackage{tikz}
\usepackage{tikz-cd}
\usepackage{centernot}
\usepackage{wasysym}
\usepackage{pgfplots}
\pgfplotsset{compat=1.18}
\usepackage{ stmaryrd }
\usepackage{math}

%Definiere Let-Command für algorithmen
\newcommand*\Let[2]{\State #1 $\gets$ #2}

\input kvmacros

%Größe der Ränder setzen
\geometry{a4paper,left=3cm, right=3cm, top=3cm, bottom=3cm}

%Kopf- und Fußzeile
\pagestyle {fancy}
\fancyhead[L]{Tutor: \TUTOR}
\fancyhead[C]{\COURSE}
\fancyhead[R]{\TODAY}

\fancyfoot[L]{}
\fancyfoot[C]{}
\fancyfoot[R]{Seite \thepage /\pageref*{LastPage}}

%Formatierung der Überschrift, hier nichts ändern
\def\header#1#2{
  \begin{center}
    {\Large Übungsblatt #1}\\
    {(Abgabetermin #2)}
  \end{center}
}

%Definition der Punktetabelle, hier nichts ändern
\newcounter{punktelistectr}
\newcounter{punkte}
\newcommand{\punkteliste}[2]{%
  \setcounter{punkte}{#2}%
  \addtocounter{punkte}{-#1}%
  \stepcounter{punkte}%<-- also punkte = m-n+1 = Anzahl Spalten[1]
  \begin{center}%
  \begin{tabularx}{\linewidth}[]{@{}*{\thepunkte}{>{\centering\arraybackslash} X|}@{}>{\centering\arraybackslash}X}
      \forloop{punktelistectr}{#1}{\value{punktelistectr} < #2 } %
      {%
        \thepunktelistectr &
      }
      #2 &  $\Sigma$ \\
      \hline
      \forloop{punktelistectr}{#1}{\value{punktelistectr} < #2 } %
      {%
        &
      } &\\
      \forloop{punktelistectr}{#1}{\value{punktelistectr} < #2 } %
      {%
        &
      } &\\
    \end{tabularx}
  \end{center}
}

\begin{document}

% Hier nichts ändern
\begin{tabularx}{\linewidth}{m{0.2 \linewidth}X}
  \begin{minipage}{\linewidth}
    \STUDENTA\\
    \STUDENTB\\
    \STUDENTC
  \end{minipage} & \begin{minipage}{\linewidth}
    \punkteliste{1}{\EXERCISES}
  \end{minipage}\\
\end{tabularx}

\header{Nr. \NUMBER}{\DEADLINE}



% ----------------------- TODO ---------------------------
% Hier werden die Aufgaben/Lösungen eingetragen:
\section*{Exercise 1}
\subsection*{1)}
Since $A=k\llbracket x\rrbracket$ is the ring of formal power series, we only need to show that it doesn't contain nonzero zero divisors.\\
Let $(a_n)_{n\in\N},(b_n)_{n\in\N}\in A$ such that $(a_n)_{n\in\N}\times(b_n)_{n\in\N}=(0,0,0,\dots)$.\\
We assume that $(a_n)_{n\in\N}\neq(0,0,0,\dots)\neq(b_n)_{n\in\N}$, so that we may prove what's to be proven by contradiction.\\
By definition of the multiplication, $(a_n)_{n\in\N}\times(b_n)_{n\in\N}=\bp{\underset{k=1}{\overset{n}{\sum}}a_kb_{n-k}}_{n\in\N}$.\\
Our proof occurs through induction.\\
\textbf{IA:}\\
Let $n=0$
\[\underset{k=0}{\overset{0}{\sum}}a_kb_{n-k}=0\implies a_0b_0=0\implies a_0=0\vee b_0=0\]
Without loss of generality we assume that $a_0=0$.\\
\textbf{IV:}
\begin{center}
    Given that $a_i=0$ with $0\leq i\leq n$ for some fixed $n\in\N$, it's to be shown that $a_{n+1}=0$.
\end{center}
\textbf{IS:}
\begin{align*}
    \underset{k=0}{\overset{n+1}{\sum}}a_kb_{n+1-k}&=0\\
    a_{n+1}b_0+\underset{k=0}{\overset{n}{\sum}}\underbrace{a_k}_{=0}b_{n+1-k}&=0\\
    a_{n+1}b_0&=0\implies a_{n+1}=0\vee b_0=0
\end{align*}
\textbf{case 1:} $a_{n+1}=0$\\
This would conclude our induction, prove that $(a_n)_{n\in\N}=(0,0,0,\dots)$, and contradict our assumption at the start.\\
\textbf{case 2:} $a_{n+1}\neq0\wedge b_0=0$\\
Then our assumption at the start implies $\exists j\in\N:\forall i\in\N:i<j:b_i=0\wedge b_j\neq0$.\\
This means $(a_n)_{n\in\N}\times(b_n)_{n\in\N}=(\dots,0,0,a_{n+1}b_j,0,0,\dots)\neq(0,0,0,\dots)$, which is also a contradiction.\\ \\
Either way, $A$ doesn't contain nonzero zero divisors, making it an integral domain.
\subsection*{2)}
Let $f=(a_n)_{n\in\N}\in A^\times$.\\
Then the inverse is $f^{-1}=(b_n)_{n\in\N}$ with $b_0=a_0^{-1}$ and $b_n=-a_0^{-1}\underset{i=1}{\overset{n}{\sum}}a_ib_{n-i}$.\\
Subsequently, we'll prove that this is a correct inverse.\\
Our proof occurs through induction.\\
\textbf{IA:}\\
Let $n=0$
\[\underset{i=0}{\overset{0}{\sum}}a_ib_{n-i}=a_0b_0=a_0a_0^{-1}=1\]
Let $n=1$
\[\underset{i=0}{\overset{1}{\sum}}a_ib_{n-i}=a_0b_1+a_1b_0=a_0(-a_0^{-1}a_1a_0^{-1})+a_1a_0^{-1}=0\]
\textbf{IV:}
\begin{center}
    Given that $\underset{i=0}{\overset{k}{\sum}}a_ib_{n-i}=0$ with $1\leq k\leq n$ for some fixed $n\in\N$, it's to be shown that $\underset{i=0}{\overset{n+1}{\sum}}a_ib_{n+1-i}=0$.
\end{center}
\textbf{IS:}
\begin{align*}
    \underset{i=0}{\overset{n+1}{\sum}}a_ib_{n+1-i}&=a_0b_{n+1}+\underset{i=1}{\overset{n+1}{\sum}}a_ib_{n+1-i}\\
    &=-a_0a_0^{-1}\underset{i=1}{\overset{n+1}{\sum}}a_ib_{n+1-i}+\underset{i=1}{\overset{n}{\sum}}a_ib_{n+1-i}\\
    &=0
\end{align*}
This means $f$ has an inverse when $a_0^{-1}$ exists, and also if $f$ has an inverse then $a_0^{-1}$ must exist.\\
Since we proved in \textbf{exercise 1 (1)} that $A$ is an integral domain, $a_0$ having an inverse is equivalent to not being equal to zero, so $A^\times=\bc{\underset{i=0}{\overset{\infty}{\sum}}a_ix^i|a_0\neq0}$ holds true.
\subsection*{3)}
Since $A^\times\subset A$, every $u\in A^\times$ can already be expressed in $A$ as $x^0u=u\in A$.\\
This means it is enough to show that every $f\in A-A^\times$ can be expressed $f=x^nu$ for a $u\in A^\times$ and $n\geq0$.\\
From \textbf{exercise 1 (2)} we know that $A-A^\times=\bc{\underset{i=0}{\overset{\infty}{\sum}}a_ix^i|a_0=0}$.\\
Let $(a_n)_{n\in\N}\in A-A^\times$.\\
Since $a_0=0$, it follows $\exists j\in\N:\forall i\in\N:i<j:a_i=0\wedge a_j\neq0$.\\
This means $(b_n)_{n\in\N}=(a_{n+j})_{n\in\N}\in A^\times$, as $b_0=a_{0+j}=a_j\neq0$.\\
Finally, $(a_n)_{n\in\N}$ can be expressed as:
\[(a_n)_{n\in\N}=\underset{i=0}{\overset{\infty}{\sum}}a_ix^i\underset{\forall i<j:a_i=0}{=}\underset{i=j}{\overset{\infty}{\sum}}a_{i}x^{i}=\underset{i=0}{\overset{\infty}{\sum}}a_{i+j}x^{i+j}=x^j\underset{i=0}{\overset{\infty}{\sum}}a_{i+j}x^{i}=x^j\underset{i=0}{\overset{\infty}{\sum}}b_ix^i=x^j(b_n)_{n\in\N}\]
\subsection*{4)}
Since every ring contains the zero ideal, we only need to show that all non-trivial ideals are of the form $(x^n)$ for $n\geq0$.\\
Let $M=\bc{f_0,f_1,\dots,f_n}\subset A$ be a non-empty subset.\\
The ideal generated by $M$ is $(M)=\bc{\underset{i=0}{\overset{n}{\sum}}\l_if_i|\l_0,\dots,\l_n\in A}$.\\
However, \textbf{exercise 1 (3)} tells us that $f_i=x^{m_i}u_i$ for $0\leq i\leq n$, $m_i\geq0$, and $u_i\in A^\times$.\\
Thus we can write $(M)=\bc{\underset{i=0}{\overset{n}{\sum}}\l_ix^{m_i}u_i|\l_0,\dots,\l_n\in A}$, and since $\l_i$ are arbitrary, we can simplify the term by setting $l\i=g_iu_i^{-1}$ for an arbitrary $g_i\in A$:
\[(M)=\bc{\underset{i=0}{\overset{n}{\sum}}g_ix^{m_i}|g_0,\dots,g_n\in A}\]
So $(M)=(\bc{x^{m_0},x^{m_1},\dots,x^{m_n}})$, which can be simplified to $(M)=(\bc{x^m})$ (for $m=\min\bc{m_0,\dots,m_n}$), since $x^m|x^{m_i}$ for $0\leq i\leq n$.
\section*{Exercise 2}
\subsection*{1)}
Since $m_P$ and $A$ are both ideals, their intersection $m_P\cap A$ is also an ideal.\\
It is to be shown that $m_P\cap A$ is maximal.\\
Let the following ring homomorphisms:
\[ev_P:\begin{matrix}
    \\
    A&\rightarrow& K\\
    f&\mapsto&f\of{P}
\end{matrix}\qquad\text{and}\qquad \pi:\begin{matrix}
    \\
    A&\rightarrow& A/m_P\cap A\\
    f&\mapsto&f+m_P\cap A
\end{matrix}\]
Then any $f\in A$ is precisely $f\in\ker\of{ev_P}$ if and only if $ev_P\of{f}=0$.\\
And since $m_P$ is the ideal, which only includes all functions with a zero evaluated at $P$, $m_P\cap A$ is the ideal, which only includes all functions in $A$ with a zero evaluated at $P$. In other words, $\ker\of{ev_P}=m_P\cap A$.\\
This means that $\pi:A\rightarrow A/\ker\of{ev_P}$.\\
By \textbf{Theorem 1.1.10 (4)} this implies that there exists a unique and injective $g:A\rightarrow A/m_P\cap A$, such that:
\[\begin{tikzcd}
    A \arrow{r}{\pi} \arrow[swap]{dr}{ev_P} & A/\ker\of{ev_P} \arrow{d}{g}\\
    & k &
\end{tikzcd}\]
induces the ring isomorphism $\underbrace{A/\ker\of{ev_P}}_{=A/m_P\cap A}\cong \underbrace{ev_P\of{A}}_{k}$.\\
Thus, $A/m_P\cap A$ is a field, and by \textbf{theorem 1.1.11 (2)}, also maximal.
\subsection*{2)}
Since $A=k\bs{x_1,\dots,x_n}$, $m\subset A$, and $m$ maximal $\overset{\textbf{theorem 1.1.11 (2)}}{\implies} A/m$ field, \textbf{corollary 1.1.17} states that $A/m$ must be isomorphic to some finitely generated $k$-algebra $B$.\\
$B$ being a finitely generated $k$-algebra means there exists a ring homomorphism $\f:k\rightarrow B$, though since $k$ and $B$ are both fields $f$ is actually a homomorphism between fields, which is the exact definition of a field extension.\\
And since $A/m\cong B$, it means $A/m$ is also a finite field extension of $k$.
\subsection*{3)}



























\section*{Exercise 3}
Let $A$ be a boolean ring and $a\in A$.
\subsection*{1)}
Then
\[2a\overset{A\text{ boolean}}{=}(2a)^2=4a^2\overset{A\text{ boolean}}{=}4a\implies0=2a\]
\subsection*{2)}
Note: $\forall a,b\in A: ab(a+b)=a^2b+ab^2=ab+ab=2ab=0$.\\
Let $\mathfrak{p}\subset A$ be a prime ideal.\\
Then $\forall a,b\not\in\mathfrak{p}:ab\not\in\mathfrak{p}$, but $ab(a+b)=0\in\mathfrak{p}$.\\
Since for all $xy\in\mathfrak{p}$ either $x\in\mathfrak{p}$ or $y\in\mathfrak{p}$, this means:
\[\forall a,b\not\in\mathfrak{p}:\underbrace{ab}_{\not\in\mathfrak{p}}(a+b)=0\in\mathfrak{p}\implies a+b\in\mathfrak{p}\]
Let $\mathfrak{p}\subsetneq I\subset A$ be an ideal.\\
Then for any $x\in I$ but $x\not\in\mathfrak{p}$:
\[\forall y\not\in\mathfrak{p}:x+y\in\mathfrak{p}\overset{\mathfrak{p}\subsetneq I}{\implies}x+y\in I\overset{(I,+)\text{ group}}{\underset{x\in I}{\implies}}x+x+y=2x+y=y\in I\]
, meaning $I=A$, so $\mathfrak{p}$ is maximal.\\ \\
Now we prove that $A/\mathfrak{p}\cong\mathbb{F}_2$.\\
Since we know $\forall a,b\not\in\mathfrak{p}:a+b\in\mathfrak{p}$, it follows that $a+a+b=b\in a+\mathfrak{p}$, meaning that $a+\mathfrak{p}$ is the set of all elements not in $\mathfrak{p}$.\\
This implies $A/\mathfrak{p}=\bc{0+\mathfrak{p},a+\mathfrak{p}}$ for an $a\not\in\mathfrak{p}$.\\
Thus, $A/\mathfrak{p}$ is a field with:
\[(0+\mathfrak{p})+(0+\mathfrak{p})=(0+\mathfrak{p})\qquad(0+\mathfrak{p})+(a+\mathfrak{p})=(a+\mathfrak{p})\qquad(a+\mathfrak{p})+(a+\mathfrak{p})=(0+\mathfrak{p})\]
and
\[(0+\mathfrak{p})(0+\mathfrak{p})=(0+\mathfrak{p})\qquad(0+\mathfrak{p})(a+\mathfrak{p})=(0+\mathfrak{p})\qquad(a+\mathfrak{p})(a+\mathfrak{p})=(a+\mathfrak{p})\]
, so $A/\mathfrak{p}\cong\mathbb{F}_2$.
\subsection*{3)}
%possible solution: $(M\cup\bc{a})=(M)$ for $\emptyset\neq M\subset A$ and $a\in A$















\section*{Exercise 4}
Let $A$ be a ring and $I,J\subseteq A$ ideals.
\subsection*{1)}
Let $b\in (I:J)$.\\
By the definition of the coset:
\[bJ\subseteq I\Longleftrightarrow \forall j\in J:bj\in I\]
And since for all $i\in I$ and $x\in A$ their product $xi\in I$, it follows that:
\[(\forall a\in A:abj\in I)\implies ab\in (I:J)\]
So for all $b\in(I:J)$ and $a\in A$ their product $ab\in(I:J)$, meaning $(I:J)$ is an ideal.
\subsection*{2)}
Let $M=\bc{\mathfrak{p}\subseteq A\text{ prime ideal}|I\subseteq\mathfrak{p},J\not\subseteq\mathfrak{p}}$.
\subsubsection*{"$\subset$":}
Let $a\in(\sqrt{I}:J)$.\\
Then $\forall j\in J:\exists m\in\Z_{>0}:(aj)^m\in I$.\\
Since $I\subset\underset{\mathfrak{p}\in M}{\bigcap}\mathfrak{p}$, it means that $(aj)^m\in\underset{\mathfrak{p}\in M}{\bigcap}\mathfrak{p}$.\\
And since $\underset{\mathfrak{p}\in M}{\bigcap}\mathfrak{p}$ is a prime ideal, it follows:
\[\underbrace{aj\cdots aj}_{m\text{ times}}\in\underset{\mathfrak{p}\in M}{\bigcap}\mathfrak{p}\implies aj\in\underset{\mathfrak{p}\in M}{\bigcap}\mathfrak{p}\implies a\in\underset{\mathfrak{p}\in M}{\bigcap}\mathfrak{p}\vee j\in \underset{\mathfrak{p}\in M}{\bigcap}\mathfrak{p}\]
However $j\in \underset{\mathfrak{p}\in M}{\bigcap}\mathfrak{p}$ would be a contradiction for some $j\in J$, since $J\not\subseteq\mathfrak{p}$ for all $\mathfrak{p}\in M$.\\
Thus $a\in\underset{\mathfrak{p}\in M}{\bigcap}\mathfrak{p}$.
\subsubsection*{"$\supset$":}
Let $a\in\underset{\mathfrak{p}\in M}{\bigcap}\mathfrak{p}$.\\
Let $j\in J$.\\
Then $aj\in\underset{\mathfrak{p}\in M}{\bigcap}\mathfrak{p}$.\\
Since $M\subseteq\bc{\mathfrak{p}\subseteq A\text{ prime ideal}|I\subseteq\mathfrak{p}}$, it follows by \textbf{corollary 1.3.9}, that:
\[\underset{\mathfrak{p}\in M}{\bigcap}\mathfrak{p}\subseteq\underset{\underset{\mathfrak{p}\text{ prime ideal}}{I\subseteq\mathfrak{p}}}{\bigcap}\mathfrak{p}=\sqrt{I}\]
So $aj\in\sqrt{I}$, meaning there exists an $m\in\Z_{>0}$ such that $(aj)^m\in I$, which directly implies $a\in(\sqrt{I}:J)$.
\subsection*{3)}
Since $k$ is algebraically closed, every polynomial in $k\bs{x_1,\dots,x_n}$ can be written as be expressed as the product of linear combinations $(x_1-a_1,\dots,x_n-a_n)$ with $a_i\in k$ for $1\leq i\leq n$.
\subsubsection*{"$\subset$":}
Let $f=\underset{k=1}{\overset{\deg\of{f}}{\prod}}(x_1-a_{k_1},\dots,x_n-a_{k_n})\in(I\of{X}:I\of{Y})$.\\
Then for all $g=\underset{k=1}{\overset{\deg\of{g}}{\prod}}(x_1-y_{k_1},\dots,x_n-y_{k_n})\in I\of{Y}:f\times g\in I\of{X}$.\\
This means $X\subseteq V\of{\bc{f\times g}}$, since $f\times g\in I\of{X}$ means all the elements in $X$ are zeroes in $f\times g$.\\
This also means that for every $h\in I\of{X\cap Y}$ with $h|g$, we have $X\subseteq V\of{\bc{f\times h}}$, since every element in $X\cap Y$ is a zero in $h$, and every element in $X-Y$ must be a zero in $f$ (since $f\times g\in I\of{X}\implies\forall x\in X-Y:f(x)=0\vee g(x)=0$, and $g(x)\neq0$ since $g\in I\of{Y}$).
However every element of $X-Y$ being a zero in $f$ means $f\in I\of{X-Y}$.
\subsubsection*{"$\supset$":}
Let $f\in I\of{X-Y}$.\\
Then $\forall x\in X-Y:f\of{x}=0$.\\
Let $g\in I\of{Y}$.\\
Then $\forall y\in Y:g\of{y}=0$.\\
This means $(\forall x\in X:x\in X-Y\vee x\in Y)\implies (f\times g)\of{x}=0$.\\
Furthermore, $\forall h\in I\of{Y}:f\times h\in I\of{X}$, which implies $f\in(I\of{X}:I\of{Y})$.
\section*{Exercise 5}
\subsection*{1)}
$V(I)$ is the set of all $a,b\in k$, for which $a^2-b^3=0$.\\
Just like the exercise suggests, we assume $k=\C$ and draw the real valued points:
\begin{center}
    \begin{tikzpicture}
        \begin{axis}[
                axis lines=middle,
                xlabel=$x$,
                ylabel=$y$,
                xmin=-0.5, xmax=6.5,
                ymin=-2.5, ymax=2.5,
                xtick={0,1,2,3,4,5,6},
                ytick={-2,-1,0,1,2},
                samples=100,
                domain=0:2,
                ]
            \addplot [thick, blue] {sqrt(x^3)};
            \addplot [thick, blue] {-sqrt(x^3)};
        \end{axis}
    \end{tikzpicture}
\end{center}
\subsection*{2)}
\subsubsection*{"injective":}
Let $a,b\in k$ such that $f\of{a}=f\of{b}$.\\
Then $(a^{-2},a^{-3})=(b^{-2},b^{-3})\implies a^{-2}=b^{-2}\wedge a^{-3}=b^{-3}$.\\
This means:
\[a=a^3a^{-2}=a^3b^{-2}=\bp{a^{-3}}^{-1}b^{-2}=\bp{b^{-3}}^{-1}b^{-2}=b^3b^{-2}=b\]
So $f\of{a}=f\of{b}\implies a=b$.
\subsubsection*{"surjective":}
Let $(x,y)\in V\of{I}-\bc{(0,0)}$.\\
Then we know that $y^2-x^3=0$, which implies $y^2=x^3$.\\
Then we choose $a\in k-\bc{0}$ as $a=x^{-\frac{1}{2}}$, which gives us $a=\bp{x^3}^{-\frac{1}{6}}=y^{-\frac{1}{3}}$.\\
It follows:
\[f\of{a}=(a^{-2},a^{-3})=\bp{\bp{x^{-\frac{1}{2}}}^{-2},\bp{y^{-\frac{1}{3}}}^{-3}}=(x,y)\]
We see that $f\of{0,0}$ is undefined, which we assign to be $f\of{0,0}=O$.\\
So every element in $V\of{I}-\bc{(0,0)}\sqcup O$ has an element in $k$, which is assigned through $f$.
\subsection*{3)}
Let $a,b\in k$ with $a\neq0\neq b$ and $a\neq b\neq -b$.\\
Then
\[y(x)=\overbrace{\frac{a^{-3}-b^{-3}}{a^{-2}-b^{-2}}}^mx-\overbrace{\frac{a^{-3}-b^{-3}}{a^{-2}-b^{-2}}a^{-2}+a^{-3}}^c\]
is the line through $f\of{a}$ and $f\of{b}$.\\
If what's to be shown is indeed correct then $f\of{a+b}=\bp{\bp{a+b}^{-2},\bp{a+b}^{-3}}\in V\of{I}$ means that $y\of{\bp{a+b}^{-2}}=-\bp{a+b}^{-3}=-\frac{1}{\bp{a+b}^3}$.\\
We calculate that:
\begin{align*}
    y\of{\bp{a+b}^{-2}}&=\frac{a^{-3}-b^{-3}}{a^{-2}-b^{-2}}\bp{a+b}^{-2}-\frac{a^{-3}-b^{-3}}{a^{-2}-b^{-2}}a^{-2}+a^{-3}\\
    &\\
    &=\frac{\frac{1}{a^3}-\frac{1}{b^3}}{\frac{1}{a^2}-\frac{1}{b^2}}\frac{1}{\bp{a+b}^2}-\frac{\frac{1}{a^3}-\frac{1}{b^3}}{\frac{1}{a^2}-\frac{1}{b^2}}\frac{1}{a^2}+\frac{1}{a^3}\\
    &\\
    &=\frac{\frac{b^3-a^3}{a^3b^3}}{\frac{b^2-a^2}{a^2b^2}}\frac{1}{\bp{a+b}^2}-\frac{\frac{b^3-a^3}{a^3b^3}}{\frac{b^2-a^2}{a^2b^2}}\frac{1}{a^2}+\frac{1}{a^3}\\
    &\\
    &=\frac{\bp{b^3-a^3}a^2b^2}{\bp{b^2-a^2}a^3b^3}\frac{1}{\bp{a+b}^2}-\frac{\bp{b^3-a^3}a^2b^2}{\bp{b^2-a^2}a^3b^3}\frac{1}{a^2}+\frac{1}{a^3}\\
    &\\
    &=\frac{a^2b^5-a^5b^2}{a^3b^5-a^5b^3}\bp{\frac{1}{\bp{a+b}^2}-\frac{1}{a^2}}+\frac{1}{a^3}\\
    &\\
    &=\frac{a^2b^5-a^5b^2}{a^3b^5-a^5b^3}\frac{-2ab-b^2}{a^4+2a^3b+a^2b^2}+\frac{1}{a^3}\\
    &\\
    &=\frac{-2a^3b^6-a^2b^7+
    2a^6b^3+a^5b^4}{2a^6b^6+a^5b^7-a^9b^3-2a^8b^4}+\frac{2a^3b^6+a^2b^7-a^6b^3-2a^5b^4}{2a^6b^6+a^5b^7-a^9b^3-2a^8b^4}\\
    &\\
    &=\frac{a^6b^3-a^5b^4}{2a^6b^6+a^5b^7-a^9b^3-2a^8b^4}\\
    &\\
    &=\frac{a-b}{2ab^3+b^4-a^4-2a^3b}=\frac{a-b}{\bp{a-b}\bp{-a^3-b^3-3a^2b-3ab^2}}=-\frac{1}{\bp{a+b}^3}\\
\end{align*}
So the interception $(c,d)$ of $V\of{I}$ and the line passing through $f\of{a},f\of{b}$ truly satisfies $f\of{a+b}=(c,-d)$.\\
For the case $a=b\neq0$:
The line through $f\of{a}$ and $f\of{b}$ must then be a tangent, and
\[f\of{a+b}=f\of{2a}=\bp{\frac{1}{4}a^{-2},\frac{1}{4}a^{-3}}\]
means its slope is exactly
\[\frac{a^{-3}-\frac{1}{8}a^{-3}}{a^{-2}-\frac{1}{4}a^{-2}}=\frac{\frac{8-1}{8a^3}}{\frac{4-1}{4a^2}}=\frac{36a^2}{24a^3}=\frac{3}{2a}\]
, which is equal to the slope of the derivative of the cusp:
\[y=\pm x^{\frac{3}{2}}\implies \frac{dy}{dx}(a)=\frac{3}{2a}\]
So the claim still holds true.\\
For the case $a=b=0$:\\
There is no interception, since there is no line through $f\of{0}$ and $f\of{0}$, so it is undefined at this point.\\
For the case $a=-b\neq0$:\\
The line through $f\of{a}$ and $f\of{b}$ is simply a vertical, meaning there is no third intersection, and $f\of{a+b}=f\of{0}$, so it is also undefined.
\end{document}

%%% Local Variables:
%%% mode: latex
%%% TeX-master: t
%%% End:
